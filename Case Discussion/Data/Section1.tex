\newpage
\section{Discussion Report}
	\subsection{Case 1}
		\subsubsection{Role Assignment}
		\begin{itemize}
  			\item Hao Sun: \emph{Moderator}
  			\item Qinhan Hou: \emph{Devil’s advocate}
  			\item Yifan Liu: \emph{Editor} 
  			\item Zezhong Zhang: \emph{Recorder}
		\end{itemize}
		\subsubsection{Standpoint of each member}
		\begin{itemize}
  			\item Hao Sun: 
  			\item Qinhan Hou:
  			\item Yifan Liu: I agree with Aisha considering that the quality of work is important, so the first thing should do is to focus on the main feature and its quality. And I also disagree with Srinivas. I think that open-source software is great for personal development but as a product, we still need time to go through the code, and it's a time-consuming thing to do so. On the other hand, this is an anti-LGTBQ+ right organization support community, so it's better not to get involved with them. 
  			And that is also the ACM Code of Ethics suggests us to do, we need to be fair and take action not to discriminate.
  			\item Zezhong Zhang: I agree with aisha's idea, because the software needs to ensure the availability and integrity of the main functionality, so that it can be completed before the deadline while ensuring a certain degree of experience and functionality. As for Javier' idea, the disadvantage is that some functions are incomplete which have an impact on user experience. Linnea's idea is good, but resetting the release time makes people lose patience and confidence about the software. So it is better to release software with main functions and then improve it step by step so that users could accept it easier. Srinivas' idea is risky, maybe using the open-source software made by the company that had support with anti-LGBT rights behavior could cause the public to have a bad impression on the product, which will be terrible when promoting. And according to ACM Code of Ethics, we should be fair and take action not to discriminate.
		\end{itemize}
		\subsubsection{Discussion regarding the questions}
        Moderator agree with Aisha, so as Editor and Recorder.\\\\
        Devil’s advocate: Looking at the story, this is the team product manager who gave a deadline request, but now the development side has feedback that it can't be done on time. This is a problem with the scheduling and an internal problem of the team. In terms of ethics, we cannnot only complete part of the development and announce that the rest will be released later on. According to ACM’s Code of Ethics, which is "Strive to achieve high quality in both the processes and products of professional work", Professionals should be cognizant of any serious negative consequences affecting any stakeholder that may result from poor quality work and should resist inducements to neglect this responsibility.\\\\
        Moderator: Since the programmers have accepted the job, they must try their best to meet the basic requirements of the customer. If they postpone the deadline, customers will even don't have a product to use.\\\\
        Devil’s advocate: The contract signed by the customer and us requires that all needs are met. Now that we only deliver part of the due demand, we still fail to meet the customer's needs. So we should postpone the deadline and finally deliver a product that satisfy the customers.
        
	\subsection{Case 2}
		\subsubsection{Role Assignment}
		\begin{itemize}
  			\item Hao Sun: \emph{Editor}
  			\item Qinhan Hou: \emph{Recorder}
  			\item Yifan Liu: \emph{Devil’s advocate}
  			\item Zezhong Zhang: \emph{Moderator}
		\end{itemize}
		\subsubsection{Standpoint of each member}
		\begin{itemize}
  			\item Hao Sun: 
  			\item Qinhan Hou:
  			\item Yifan Liu: 
  			\item Zezhong Zhang: In our culture, we seek common ground while reserving differences and try to understand each other when we meet with conflict. And in conversation, People with high positions lead the conversation, but if members are equal, the pause and so on may vary according to the personality of each person. About the case, Javier relies on seniority and has a dominant character. He wants to take the lead in the direction of the conversation and he thinks that voting weight should vary according to experience, and I also agree with voting weight. Aisha speaks bluntly and values most people’s ideas, I like her actions. Linnea don't know how to express her ideas and thinks that we should gain consensus before making decisions which I think it's hard to gain. Srinivas is undecided, and I think voting could make things easier. 
		\end{itemize}
		\subsubsection{Discussion regarding the questions}