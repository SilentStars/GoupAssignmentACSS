\newpage
\section{Discussion Report}
	\subsection{Case 1}
		\subsubsection{Role Assignment}
		\begin{itemize}
  			\item Hao Sun: \emph{Moderator}
  			\item Qinhan Hou: \emph{Devil’s advocate}
  			\item Yifan Liu: \emph{Recorder} 
  			\item Zezhong Zhang: \emph{Recorder}
		\end{itemize}
		\subsubsection{Standpoint of each member}
		\begin{itemize}
  			\item Hao Sun: Agree with Qinhan. The client, as a customer, who paid us for the software. This means that we should satisfy our customer's needs. If we postponed the release date, it might cause economic losses to our customers. On the other hand, it violates sustainable development since it will waste our clients' time and energy.
  			\item Qinhan Hou: According to the case, this is the team's product manager who gave a deadline to deliver the software. However, now the development side's feedback said that it could not be done on time. That is a scheduling issue, and it is a problem within the group. Morally speaking, we can not only do part of the project before the released date and then hold a conference to explain the rest of the content will be released in the next version. According to the ACM Code of Ethics: "Maintain high standards of professional competence, conduct, and ethical practice."
  			\item Yifan Liu: I agree with Aisha considering that the quality of work is important, so the first thing should do is to focus on the main feature and its quality. And I also disagree with Srinivas. I think that open-source software is great for personal development but as a product, we still need time to go through the code, and it's a time-consuming thing to do so. On the other hand, this is an anti-LGTBQ+ right organization support community, so it's better not to get involved with them. 
  			And that is also the ACM Code of Ethics suggests us to do, we need to be fair and take action not to discriminate.
  			\item Zezhong Zhang: I agree with aisha's idea, because the software needs to ensure the availability and integrity of the main functionality, so that it can be completed before the deadline while ensuring a certain degree of experience and functionality. As for Javier' idea, the disadvantage is that some functions are incomplete which have an impact on user experience. Linnea's idea is good, but resetting the release time makes people lose patience and confidence about the software. So it is better to release software with main functions and then improve it step by step so that users could accept it easier. Srinivas' idea is risky, maybe using the open-source software made by the company that had support with anti-LGBT rights behavior could cause the public to have a bad impression on the product, which will be terrible when promoting. And according to ACM Code of Ethics, we should be fair and take action not to discriminate.
		\end{itemize}
		\subsubsection{Discussion regarding the questions}
        Moderator agree with Aisha, so as Editor and Recorder.\\\\
        Devil’s advocate: Looking at the story, this is the team product manager who gave a deadline request, but now the development side has feedback that it can't be done on time. This is a problem with the scheduling and an internal problem of the team. In terms of ethics, we cannnot only complete part of the development and announce that the rest will be released later on. According to ACM’s Code of Ethics, which is "Strive to achieve high quality in both the processes and products of professional work", Professionals should be cognizant of any serious negative consequences affecting any stakeholder that may result from poor quality work and should resist inducements to neglect this responsibility.\\\\
        Moderator: Since the programmers have accepted the job, they must try their best to meet the basic requirements of the customer. If they postpone the deadline, customers will even don't have a product to use.\\\\
        Devil’s advocate: The contract signed by the customer and us requires that all needs are met. Now that we only deliver part of the due demand, we still fail to meet the customer's needs. So we should postpone the deadline and finally deliver a product that satisfy the customers.
        
	\subsection{Case 2}
		\subsubsection{Role Assignment}
		\begin{itemize}
  			\item Hao Sun: \emph{Recorder}
  			\item Qinhan Hou: \emph{Recorder}
  			\item Yifan Liu: \emph{Devil’s advocate}
  			\item Zezhong Zhang: \emph{Moderator}
		\end{itemize}
		\subsubsection{Standpoint of each member}
		\begin{itemize}
  			\item Hao Sun: I agree with Qinhan. The weighted voting system will create unfairness and add individual bias to the conclusion. Humans make mistakes, so we should always try to listen to others' opinions and try not to be trapped by our limited cognition.
  			\item Qinhan Hou: I think Javier is the one who causes the issue because he carried out much emotion during the meeting. Moreover, he said that not all votes should weigh the same; it will cause a group's disorientation.  As for Srinivas, she should learn how to speak out her mind. Like Aisha, she is doing well with that and reacts fast. Linea and Aisha's divergence is expected. It is always hard for a group to reach consensus, so a vote is always the best solution.
  			\item Yifan Liu: My point is that in some critical cases, we do need some weight depends on each member's experience or something else to make the vote work, just like the voting system of the company's board of directors. So I agree with Javier. As for Linnea and Srinivas, I think they did not bring the discussion's value into play. The meeting is for us to solve the conflicts and problems of a group. If you are undecided, it means that there is still some disagreement within your mind. In such a case, you should make yourself clear.
  			\item Zezhong Zhang: In our culture, we seek common ground while reserving differences and try to understand each other when we meet with conflict. And in conversation, People with high positions lead the conversation, but if members are equal, the pause and so on may vary according to the personality of each person. About the case, Javier relies on seniority and has a dominant character. He wants to take the lead in the direction of the conversation and he thinks that voting weight should vary according to experience, and I also agree with voting weight. Aisha speaks bluntly and values most people’s ideas, I like her actions. Linnea don't know how to express her ideas and thinks that we should gain consensus before making decisions which I think it's hard to gain. Srinivas is undecided, and I think voting could make things easier. 
		\end{itemize}
		\subsubsection{Discussion regarding the questions}
		Moderator thinks the conflicts approached in our culture are often by misunderstandings and conflicts.  We all agree with that.\\\\
		As for the conversations, it is not polite to interrupt while others are speaking, and we think in our culture, the pace of conversation is carried out by the "ending" of our statement. Like when your statement is over, you can say "It's your turn." or "How do you think?" to give out the turn. \\\\
		As for question 3 and 4, there is some different thought within our group:\\\\
		For Javier, Yifan and Zezhong think it is ok to use a weighted voting system, and there should be a leader who has the most experience. However, Hao and Qinhan against that: This will narrows the team's sight and add bias to the project. On the other hand, the experience is not always right because the world is changing rapidly; a group should use every member's mind. Diversity is great at this point. But finally, we agreed that a weighted voting system might mislead the group and use a regular vote to decide.\\\\
		As for Srinivas, we all agree that she needs to improve the skill of representing herself.\\\\
		Aisha is the one we all support. She responds quickly and speaks her self out if there is some doubt. And we all agree to propose a vote to decide which way the group is going.\\\\
		As for Linnea, we think it is okay not to interrupt someone, but in that case, she should hand up or show some sign to her teammates that she has something to say. Otherwise, it will end up with a knot in her mind that she had not shared her's opinion.
		
		