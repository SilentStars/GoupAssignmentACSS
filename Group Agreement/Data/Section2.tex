\newpage
\section{Reflection on group work}
%The structure may be like..
%\begin{itemize}
%    \item What this reflection about
%    \item Explore the benefits of group work
%    \item Explore the challenge of group work
%    \item Give some example of challenge that we have faced or the benefits we have got
%    \item How we handling challenges
%    \item What we should perform next time
%\end{itemize}
\subsection{The benefits of group work}
Different people may have a different perspective, working with yourself may some time cause a lopsided view. Brainstorming with a group of people can always output a great idea.
\subsection{The challenge of group work}
Just like the benefits, the challenge also comes from the difference of individual. Sometimes we might encounter some confliction, or maybe the group is acting passively and wasting time while meeting. So a group must have a clear goal and support each other.
\subsection{The challenge our group have met}
For each of us, online teaching and study abord is a new thing. We need to adapt to it and came to a methodology while each member may have a different time zone and have other works to be done. And that's the most challenging issue we have.
\subsection{How we address those challenge}
First, by implementing an agreement. Then, we clarify our duty within the group and distribute the workload.
\subsection{How we can manage to make group agreement work}
\begin{itemize}
    \item As for participating fully (in spirit and actuality) and professionally, we should attend every meeting on time and emphasize that everyone must do their job independently and utilize their technical skills.
    \item As for the rule "Each member should abiding the rules of academic honesty", if anyone cheated or disobeyed academic honesty, we will warn for the first time and even repel him if the situation is serious.
    \item As for "Meeting responsibilities", we will supervise each other and help each other to make point short and clear.
    \item As for the rule "Respect each other", we are from the same culture background and we will keep silent when someone is talking.
\end{itemize}
\subsection{How we plan to prepare}
Firstly, we read carefully about the given information and  focus on the problems we ought to cope with. Secondly, we arrange a period of time for conversation. Finally, we talk about the questions and distribute the workload of each other.
\subsection{How do we plan to communicate}
The tool we use for communication is WeChat. Usually, we arrange a period of time when we are all free and start conversation. 
\subsection{How we distribute the workload}
We usually split the task into several parts according to people and then randomly choose a part for each of us.
