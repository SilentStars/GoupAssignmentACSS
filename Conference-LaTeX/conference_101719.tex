\documentclass[conference]{IEEEtran}
\IEEEoverridecommandlockouts
% The preceding line is only needed to identify funding in the first footnote. If that is unneeded, please comment it out.
\usepackage{cite}
\usepackage{amsmath,amssymb,amsfonts}
\usepackage{algorithmic}
\usepackage{graphicx}
\usepackage{textcomp}
\usepackage{xcolor}
\renewcommand\refname{}
\def\BibTeX{{\rm B\kern-.05em{\sc i\kern-.025em b}\kern-.08em
    T\kern-.1667em\lower.7ex\hbox{E}\kern-.125emX}}
\begin{document}

\title{Report of professional competencies research\\
%{\footnotesize \textsuperscript{*}Note: Sub-titles are not captured in Xplore and
%should not be used}
%\thanks{Identify applicable funding agency here. If none, delete this.}
}

\author{\IEEEauthorblockN{Hao Sun}
\IEEEauthorblockA{\textit{Group No.8}}
\and
\IEEEauthorblockN{Qinhan Hou}
\IEEEauthorblockA{\textit{Group No.8}}
\and
\IEEEauthorblockN{Yifan Liu}
\IEEEauthorblockA{\textit{Group No.8}}
\and
\IEEEauthorblockN{Zezhong Zheng}
\IEEEauthorblockA{\textit{Group No.8}}
}

\maketitle

\begin{abstract}
This document is the report of professional competencies research towards college students, mainly target master students that study in computer science.The topic of this report is which professional competencies do college students think are important for a computer scientist. We choose this topic in order to study a series of questions, like how college students think about what is the critical competencies for a computer scientist. We surveyed some college students by questionnaire tools, and used those data to back up this report. 
\end{abstract}

\begin{IEEEkeywords}
computer science, professional competencies, college students
\end{IEEEkeywords}

\section{Introduction}
\subsection{topic}
\par The target group is college students, the majority are computer related students, but not limited to it.
\par The professional competencies we studied is not limited to technical
competencies.

\subsection{background}
\par In recent years, the development of computer science is rapidly, and computers has been more and more closely integrated with all aspects of society, and there are more people begun to choose to study computer science and be programmers as their future careers.
\par Courses in college help students develop professional competencies in engineering education, including discipline knowledge, thinking skills, technology skills, professional skills and so on.
\subsection{why}
\par We did this survey in order to 
\begin{enumerate}
    \item See how college students think about what is the critical competencies for a computer scientist
    \item See what skills mentioned above are cultivated in college
    \item What makes a computer scientist be a computer scientist
    \item Compare views from different groups in different fields, like students and
    employers
    \item Think how computer science students can develop their skills to a higher level
\end{enumerate}
\par For college students, it is an excellent way to determine the direction of the study and what's the competencies needed for higher level of programming.

\section{Method}
We chose to survey since we want to access more data and see what is on students' heads. Furthermore, to have an overview of the situation.
We chose two tools, Google Forms and a Chinese questionnaire platform WenJuanXing, to perform the survey, collect responses and analyze the collected information. Google Forms is commonly used survey tool, and we choose it to build and receive data quickly. As for the WenJuanXing, it is a Chinese website that focuses on the questionnaire and has a lot of data processing functions that are pretty helpful.

\section{Result}

Here are the main questions we used in the survey below.
\begin{enumerate}
    \item How would you ranking those fundamental professional competencies below?(for a computer scientist) *
    \item How would you ranking those core professional competencies below? (for a computer scientist) *
    \item What do you think is the ability that distinguishes computer scientists from ordinary programmers? *
    \item What professional competencies do you think are most important for a computer scientist?
    \item Why do you think these professional competencies are most important?
    
\end{enumerate}
\par The expected target of our survey is 30 people. So far, we have received 24 people’s answers to the questionnaire, which exceeded our expectations and the overall proportion reached 80 percent.
\par Below we analyze the answers to the main questions.
\subsection{QUESTION ONE}
As for question one "How would you ranking those fundamental professional competencies below?", we have six choices in total for people to rank from. These choices are listed below.
\begin{enumerate}
    \item Self-awareness
    \item Self-management
    \item Responsible decision-making
    \item Social awareness
    \item Relationship skills
    \item Technical skills
\end{enumerate}
\par From the result, we calculated the average scores of these fundamental professional competencies,4.25 scores for technical skills, 3.67 scores for relationship skills, 3.5 scores for self-management, 3.33 scores for responsible decision-making, 2.58 scores for self-awareness and 1.92 for social awareness.
\par So mostly people think the most significant fundamental professional competencies for a computer scientist is technical skills and the least significant one is social awareness.
\subsection{QUESTION TWO}
\par As for question two "How would you ranking those core professional competencies below?", we have six choices in total for people to rank from. These choices are listed below.
\begin{enumerate}
    \item Math skills
    \item Alogrithm skills
    \item Creativity
    \item Cross-domain interoperability
    \item Cooperation skill
    \item Others
\end{enumerate}
\par We did some calculations to the average scores of these professional competencies. So the result is 3.17 scores for Alogrithm skills and Creativity, 3.08 scores for Math skills, 2.08 scores for Cross-domain interoperability and Cooperation skill, 0 for others.
\par We can see from the result that most people think Creativity and Alogrithm skills are more important than other professional competencies in views of college students.

\subsection{QUESTION THREE}
\par For the question "What do you think is the ability that distinguishes computer scientists from ordinary programmers?", we offered some answers as below.
\begin{enumerate}
    \item Math skills
    \item Research Capacity
    \item Creativity
    \item Comprehension level of computer knowledge
    \item Others
\end{enumerate}
\par After calculation, we found that 8 people support Math skills, 4 people support Research Capacity and Creativity, 7 people support Comprehension level of computer knowledge and only one people think it's the problem solving ability that distinguishes computer scientists from ordinary programmers.

\subsection{QUESTION FOUR}
\par As for the question "What professional competencies do you think are most important for a computer scientist?", we received diverse answers. But the most frequent answers are research capacity and algorithm capabilities, which is similar to the most important core professional competencies in the question two.

\subsection{QUESTION FIVE}
\par As for the question "Why do you think these professional competencies are most important?", the answers we received showed that these professional competencies make scientists being scientists, which means the competencies can help them create new knowledge and boost the development of our society which are closely integrated with computer science.

\section{Discussion and future work}
\par Since our target group is mainly Chinese master student, based on what we have seen in the results, we agree that the most significant competencies for a computer scientist are technical skills, research capacity and creativity. Other competencies like social awareness are much less important.
\par In fact, this result is quite different from what employers expected. In his paper, Abelha\cite{b1} investigated some of the important competencies expected by employers, in which soft power is more important than professional core competencies. The so-called soft power includes teamwork, communication and language skills. The results of this questionnaire reflect the disconnect between ability training and social needs in the school education process. Considering that the group targeted by this survey is school students, everyone generally puts the soft power mentioned above at a very low priority, which to a certain extent leads to the problem of employment difficulties. Therefore, how to change the concept of training to improve the soft power of students and how to evaluate the soft power in education are all problems that need to be solved in the process of computer science education.
\par If we want to start a new one study based on it, we may chose to study why people think social awareness is least important. We may assume some extreme conditions about a computer scientist without social awareness and see how people would think of it.
\par According to another paper about competencies required for developing computer and information systems, employers also desire the competencies of project management, requirements management, technological support, and customer support\cite{b2}, which gives us more perspective on this question. The audience of our survey is mostly students or employees. It can be seen that even different groups of people in the same field will have different interpretations of the professional competencies required by computer scientists.

\section{References}
\begin{thebibliography}{}
\bibitem{b1} Abelha, M.; Fernandes, S.; Mesquita, D.; Seabra, F.; Ferreira-Oliveira, A.T. Graduate Employability and Competence Development in Higher Education—A Systematic Literature Review Using PRISMA. Sustainability 2020, 12, 5900.
\bibitem{b2} J. C. Nwokeji, R. Stachel, T. Holmes and R. O. Orji, "Competencies Required for Developing Computer and Information Systems Curriculum," 2019 IEEE Frontiers in Education Conference (FIE), Covington, KY, USA, 2019, pp. 1-9, doi: 10.1109/FIE43999.2019.9028613.

\end{thebibliography}


\section{Appendix:work distribution}
\par Hao Sun: Survey structure design; Survey distribution and information collection; Literature reading and investigation.
\par Qinhan Hou: Survey content design; Survey statistics and data analysis; Literature reading and investigation.
\par Yifan Liu: Survey content design; Survey distribution and information collection; Report editing.
\par Zezhong Zheng: Survey content design; Survey distribution; Survey statistics and data analysis; Report editing.

\end{document}
